\documentclass[11pt]{article}


\pdfpagewidth 8.5in
\pdfpageheight 11in

\setlength\topmargin{0in}
\setlength\headheight{0in}
\setlength\headsep{0.4in}
\setlength\textheight{8in}
\setlength\textwidth{6in}
\setlength\oddsidemargin{0in}
\setlength\evensidemargin{0in}
\setlength\parindent{0.25in}
\setlength\parskip{0.1in}

\usepackage{amssymb}
\usepackage{amsfonts}
\usepackage{amsmath}
\usepackage{mathtools}
\usepackage{amsthm}

\usepackage{fancyhdr}

\usepackage{enumerate}

      \theoremstyle{plain}
      \newtheorem{theorem}{Theorem}
      \newtheorem{lemma}[theorem]{Lemma}
      \newtheorem{corollary}[theorem]{Corollary}
      \newtheorem{proposition}[theorem]{Proposition}
      \newtheorem{conjecture}[theorem]{Conjecture}
      \newtheorem{question}[theorem]{Question}

      \theoremstyle{definition}
      \newtheorem{definition}[theorem]{Definition}
      \newtheorem{example}[theorem]{Example}
      \newtheorem{game}[theorem]{Game}

      \theoremstyle{remark}
      \newtheorem{remark}[theorem]{Remark}



\pagestyle{fancy}
\renewcommand{\headrulewidth}{0.5pt}
\renewcommand{\footrulewidth}{0pt}
\lfoot{\small \jobname{} -- Updated on \today}
\chead{\small Dr. Clontz -- Fall 2015}
\rfoot{\thepage}
\cfoot{}

\newcommand{\vect}[1]{\mathbf{#1}}
\newcommand{\veci}{\vect i}
\newcommand{\vecj}{\vect j}
\newcommand{\veck}{\vect k}
\newcommand{\<}{\langle}
\renewcommand{\>}{\rangle}
\newcommand{\Arctan}{\textrm{Arctan}}
\newcommand{\p}{\partial}
\newcommand{\mb}{\mathbb}
\newcommand{\sgn}{\textrm{sgn}}
\renewcommand{\div}{\textrm{div}\,}
\newcommand{\curl}{\textrm{curl}\,}
\newcommand{\scurl}{\textrm{scurl}\,}
\newcommand{\dvar}{\,d}

\renewcommand{\labelitemii}{\tiny$\blacksquare$}

\begin{document}

\noindent\textbf{
  MATH 1121 (Calculus for Engineering Technology) Course Outline
}

\section*{1.3 Rectangular Coordinates}

\begin{itemize}
\item Illustrate the following concepts:
  \begin{itemize}
    \item rectangular coordinate system,
    \item \(x\)-axis,
    \item \(y\)-axis,
    \item origin,
    \item quadrants,
    \item coordinates
  \end{itemize}

\item Examples:
  \begin{itemize}
    \item (Example 1) Plot \(A=(2,1)\) and \(B=(-4,-3)\).
    \item (Example 3) Three verticies of a rectangle are
          \(A=(-3,-2)\), \(B=(4,-2)\), \(C=(4,1)\). What is the
          fourth vertex?
  \end{itemize}

\item\textit{
  HW: 1-9, 15-16, 21-24
}
\end{itemize}

\section*{2.1 Some Basic Definitions}

\begin{itemize}
\item Distance Formula
  \begin{itemize}
    \item \(d=\sqrt{(x_2-x_1)^2+(y_2-y_1)^2}\)
    \item (Example 2) Find the distance between \((3,-1)\) and \((-2,-5)\).
  \end{itemize}
\item Slope Formula
  \begin{itemize}
    \item \(m=\frac{y_2-y_1}{x_2-x_1}\)
    \item (Example 3) Find the slope of the line joining
          \((3,-5)\), \((-2,-6)\).
    \item (Example 4) Find the slope of the line joining
          \((3,4)\), \((4,-6)\).
    \item \(m=\tan\theta\)
    \item (Example) Find the slope of the line with inclination \(120^\circ\).
  \end{itemize}
\item Identify parallel/perpendicular lines by slopes.
  \begin{itemize}
    \item Parallel: \(m_1=m_2\)
    \item Perpendicular: \(m_1=-\frac{1}{m_2}\)
    \item (Example 7) Prove that the triangle with vertices
          \(A=(-5,3)\), \(B=(6,0)\), and \(C=(5,5)\) is a right triangle.
  \end{itemize}
\item\textit{
  HW: 1-20, 29-36
}
\end{itemize}

\section*{2.2 The Straight Line}

\begin{itemize}
\item Point-slope form
  \begin{itemize}
    \item \(y-y_1=m(x-x_1)\)
    \item (Example 2) Find the equation of the line passing through
          \((2,-1)\) and \((6,2)\).
  \end{itemize}
\item Slope-intercept form
  \begin{itemize}
    \item \(y=mx+b\)
    \item (Example 4) Find the slope and \(y\)-intercept of the straight line
          with equation \(2y+4x-5=0\).
  \end{itemize}
\item\textit{
  HW: 1-21, 33-40
}
\end{itemize}

\section*{2.3 The Circle}

\begin{itemize}
  \item Definition
    \begin{itemize}
      \item A circle is a collection of points equidistant from its center.
    \end{itemize}
  \item Standard form
    \begin{itemize}
      \item \((x-h)^2+(y-k)^2=r^2\)
      \item (Example 1) Sketch \((x-1)^2+(y+2)^2=16\).
      \item (Example 2) Find an equation for the circle with center \((2,1)\)
            which passes through \((4,8)\).
    \end{itemize}
  \item General form
    \begin{itemize}
      \item \(x^2+y^2+Dx+Ey+F=0\)
      \item (Example 4) Find the center and radius of the circle
            \(x^2+y^2-6x+8y-24=0\).
      \item (Example) Find two functions whose graphs represent the circle
            with the previous equation.
    \end{itemize}
  \item\textit{
    HW: 1-32, 37-38
  }
\end{itemize}

\section*{2.4 The Parabola}

\begin{itemize}
  \item Definition
    \begin{itemize}
      \item A parabola is a colleciton of points equidistant from a focus point
        and a directrix line.
      \item The vertex of a parabola is the point closest to the focus and
        directix.
      \item (Example 6) Find an equation for the parabola with focus \((2,3)\)
        and directrix \((y=-1)\)
    \end{itemize}
  \item Standard forms with vertex at origin and horizontal/vertical directrix
  %TODO simplify to two equations
    \begin{itemize}
      \item \(y^2=4px\) with directrix at \(x=-p\) and focus at \((p,0)\)
      \item \(y^2=-4px\) with directrix at \(x=p\) and focus at \((-p,0)\)
      \item \(x^2=4py\) with directrix at \(y=-p\) and focus at \((0,p)\)
      \item \(x^2=-4py\) with directrix at \(y=p\) and focus at \((0,-p)\)
      \item (Example 2) Find an equation for the parabola with focus
            \((-2,0)\) and directrix \((x=2)\).
      \item (Example 4) Find the focus and directrix of the parabola with
            equation \(2x^2=-9y\).
    \end{itemize}
  \item\textit{
    HW: 1-22, 25-28
  }
\end{itemize}

\section*{2.5 The Ellipse}

\begin{itemize}
  \item Definition
    \begin{itemize}
      \item An ellipse is a collection of points where the sum of distances
            from two fixed points (called foci) is kept constant.
      \item The two points furthest/closest apart from each other on an ellipse
            are the endpoints of the major/minor axis.
      \item The sum of distances
            between each point and the foci is the same as the length of
            the major axis. The major axis passes through both foci.
    \end{itemize}
  \item Standard form with center at the origin
    \begin{itemize}
      \item \(\frac{x^2}{a^2}+\frac{y^2}{b^2}=1\), with foci given by
            \((c,0),(-c,0)\), where
            \(a^2-b^2=c^2\)
      \item \(\frac{y^2}{a^2}+\frac{x^2}{b^2}=1\), with foci given by
            \((0,c),(0,-c)\), where
            \(a^2-b^2=c^2\)
      \item (Example 3) Sketch the ellipse with equation \(4x^2+16y^2=64\),
            and compute the locations of its foci.
      \item (Example 5) Find the equation of the ellipse centered at the origin
            with an end of its minor axis at \((2,0)\) and containing the point
            \((-1,\sqrt 6)\).
    \end{itemize}
  \item\textit{
    HW: 1-26
  }
\end{itemize}

\section*{2.6 The Hyperbola}

\begin{itemize}
  \item Definition
    \begin{itemize}
      \item A hyperbola is a collection of points where the difference of distances
            from two fixed points (called foci) is kept constant.
      \item Hyperbolas are split into two curves. The two closest points
            on opposite curves are called vertices and give the transverse axis.
    \end{itemize}
  \item Standard form with center at the origin
    \begin{itemize}
      \item \(\frac{x^2}{a^2}-\frac{y^2}{b^2}=1\), with foci given by
            \((c,0),(-c,0)\) and asymptotes \(y=\pm\frac{bx}{a}\),
            where \(a^2+b^2=c^2\)
      \item  \(\frac{y^2}{a^2}-\frac{x^2}{b^2}=1\), with foci given by
            \((0,c),(0,-c)\) and asymptotes \(x=\pm\frac{by}{a}\),
            where \(a^2+b^2=c^2\)
      \item (Example 2) Sketch \(\frac{y^2}{4}-\frac{x^2}{16}=1\), labeling
            its vertices, asymptotes, and foci.
      \item (Example 3) Sketch \(4x^2-9y^2=36\), labeling
            its vertices, asymptotes, and foci.
    \end{itemize}
  \item Hypberbola with coordinate axis asymptotes
    % TODO: change c^2 to a^2
    \begin{itemize}
      \item \(xy=c^2\), with vertices given by \((c,c)\) and \((-c,-c)\)
      \item \(xy=-c^2\), with vertices given by \((c,-c)\) and \((-c,c)\)
      \item (Example 5) Sketch \(xy=4\).
    \end{itemize}
  \item\textit{
    HW: 1-14, 17-24
  }
\end{itemize}

\section*{2.7 Translation of Axes}

\begin{itemize}
  \item Vertical/horizontal translation:
    \begin{itemize}
      \item Shift right \(h\): replace \(x\) with \(x-h\).
      \item Shift up \(k\): replace \(y\) with \(y-k\).
      \item (Example 1) Give an equation of the parabola with vertex \((2,4)\)
            and focus \((4,4)\).
      \item (Example 2) Sketch the curve with equation
            \(\frac{(x-3)^2}{25}+\frac{(y+2)^2}{9}=1\).
    \end{itemize}
  \item\textit{
    HW: 1-36
  }
\end{itemize}

\section*{1.2 Algebraic Functions}

\begin{itemize}
\item Definition of a function \(y=f(x)\).
\item Types of functions
  \begin{itemize}
    \item Polynomials \(P(x)=a_0+a_1x+\dots+a_nx^n\)
    \item Rational functions \(R(x)=\frac{P(x)}{Q(x)}\) for polynomials \(P,Q\)
    \item (Example 1) Voltage equals current multiplied by resistance. If the
          voltage at time \(t\) is given by \(E(t)=2t^2=y+5\) and the
          resistance at time \(t\) is given by \(R(t)=3t+20\), then find
          a function \(I(t)\) which measures the current at time \(t\).
          Identify it as a polynomial and/or rational function.
  \end{itemize}
\item Combinations of functions
  \begin{itemize}
    \item Addition/Subtraction/Multiplication/Division
    \item Compositions \(f\circ g\) and \(g\circ f\)
    \item (Example 2) Express \(f+g\), \(f\circ g\), and
          \(g\circ f\) for the functions
          given by \(f(x)=2x^2-3\) and \(g(x)=\sqrt{x+2}\).
  \end{itemize}
\item Domain/Range
  \begin{itemize}
    \item The domain of a function is all real numbers which may be plugged
          into it without causing division by zero, even roots of negatives,
          or any other undefined operations.
    \item The range of a function is all real numbers which may possibly
          be attained by the function.
    \item (Example 5) Find the domain and range of \(f(x)=x^2+2\) and
          \(g(t)=\frac{1}{t+2}\).
    \item (Example 7) Find the domain of \(f(x)=16\sqrt x+\frac{1}{x}\).
  \end{itemize}
\item Piecewise functions
  \begin{itemize}
    \item Piecewise functions are defined differently for different parts of
          their domains.
    \item (Example 9) Find the domain for
          \[ f(t) =
          \begin{cases}
            8-2t & 0\leq t\leq 4 \\
            0 & t>4
          \end{cases}
          \]
          and compute \(f(3),f(6),f(-1)\) if possible.
  \end{itemize}
\item Exponent laws
  \begin{itemize}
    \item \(a^ma^n=a^{m+n}\)
    \item \(\frac{a^m}{a^n}=a^{m-n}\)
    \item \((a^m)^n=a^{mn}\)
    \item \((ab)^m=a^mb^m\)
    \item \((\frac{a}{b})^m=\frac{a^m}{b^m}\)
    \item \(a^0=1\)
    \item \(a^{-n}=\frac{1}{a^n}\)
    \item \(a^{1/n}=\sqrt[n]{a}\)
    \item Note \(\sqrt{a^2}=|a|\) but \(\sqrt[3]{a^3}=a\)
    \item (Example 4) Simplify
          \[
            f(x)
              =
            \frac{
              (3x^2-1)^{1/3}(2x)-(2x^3)(3x^2-1)^{-2/3}
            }{
              (3x^2-1)^{2/3}
            }
          \]
  \end{itemize}
  \item\textit{
    HW: 1-18, 21-34
  }
\end{itemize}

\section*{1.4 The Graph of a Function}

\begin{itemize}
\item Definition
  \begin{itemize}
    \item The graph of a function is the collection of all ordered pairs
      \((x,y)\) such that \(y=f(x)\)
    \item Graphing Method 1: using Chapter 2
    \item Graphing Method 2: using \(xy\) chart
    \item Vertical line test: the graph of any function hits every vertical
      line at most once
  \end{itemize}
\item Examples
  \begin{itemize}
  \item (Example 1) Graph \(f(x)=3x-5\).
  \item (Example 3) Graph \(f(x)=1+\frac{1}{x}\).
  \item (Example 4) Graph \(f(x)=\sqrt{x+1}\).
  \item (Example 6) Graph
    \[
      f(x)=
        \begin{cases}
          2x+1 & x\leq 1 \\
          6-x^2 & x>1
        \end{cases}
    \]
  \end{itemize}

\item\textit{
  HW: 1-12, 37-40
}
\end{itemize}

\section*{3.1 Limits}

\begin{itemize}
\item Limits
  \begin{itemize}
    \item \(\lim_{x\to a}f(x)=L\) means that the value of \(f(x)\) approaches
          \(L\) as the value of \(x\) approaches \(a\) in the domain of \(f\).
    \item (Example) Given \(f(x)=x^2\), we may write the following chart of
          values

      \begin{tabular}{c|c}
        \(x\) & \(f(x)\) \\\hline
        \(1.9\) & \(3.61\) \\
        \(1.99\) & \(3.9601\) \\
        \(1.999\) & \(3.996001\) \\
        \(2.001\) & \(4.004001\) \\
        \(2.01\) & \(4.0401\) \\
        \(2.1\) & \(4.41\)
      \end{tabular}

          to infer that \(\lim_{x\to 2} f(x)=4\).
    \item (Example) Given
      \[
        g(x)
          =
        \left\{
        \begin{matrix}
          x^2 & x\not= 2 \\
          -5 & x=2
        \end{matrix}
        \right.
      \]
          we have the same chart of values as before, so we assume
          \(\lim_{x\to 2} g(x)=4\).
    \item (Example) Since \(h(x)=\frac{x^3-2x^2}{x-2}\) equals \(x^2\) for
      all values of \(x\) except \(2\), we have the same chart of values
      as before, and we assume \(\lim_{x\to 2} h(x)=4\).
    \item (Example) By graphing \(y=f(x)\), \(y=g(x)\), and \(y=h(x)\),
      we can see that the points on the graph approach the point \((2,4)\)
      in all three cases.
  \end{itemize}
\item Continuity
  \begin{itemize}
    \item A continuous function satsifies the equality \(f(a)=\lim_{x\to a}f(x)\)
    for all numbers \(a\) in its domain. (The ``just plug it in'' rule.)
    \item Intuitively: the graph of the function can be drawn without lifting
    your pencil on the intervals where it is defined
    \item FACT: \(f(x)=x\) is continuous, and any combination of continuous
    functions using \(+\), \(-\), \(\times\), \(/\), \(\circ,\) or powers is
    continuous (where it is defined).
    \item (Example 3) \(f(x)=\frac{1}{x-2}\) is continuous for its entire
    domain, but undefined at its asymptote \(x=2\).
    \item (Example 5) By graphing
      \[
        \begin{matrix}
        f(x)
          =
        \left\{
        \begin{matrix}
          x+2 & x < 1 \\
          -\frac{x}{2}+5 & x\geq 1
        \end{matrix}
        \right.
          &
        g(x)
          =
        \left\{
        \begin{matrix}
          2x-1 & x\leq 2 \\
          -x+5 & x>2
        \end{matrix}
        \right.
        \end{matrix}
      \]
    we see that \(f\) is continuous except for when \(x=1\), and
    \(g\) is continuous everywhere.
  \end{itemize}
\item Limits to \(\pm\infty\)
  \begin{itemize}
    \item \(\lim_{x\to \infty}f(x)=L\) means that the value of \(f(x)\) approaches
          \(L\) as the value of \(x\) attains arbitrarily large postive values.
    \item \(\lim_{x\to -\infty}f(x)=L\) means that the value of \(f(x)\) approaches
          \(L\) as the value of \(x\) attains arbitrarily large negative values.
    \item (Example) Use a chart of values to infer that
          \(\lim_{x\to\pm\infty}\frac{1}{x}=0\).
    \item (Example 14) Use a chart of values and algebraic manipulation to
          show that \(\lim_{x\to\pm\infty}\frac{x^2+1}{2x^2+3}=\frac{1}{2}\).
  \end{itemize}
\item Evaluating limits analytically
  \begin{itemize}
    \item For continuous functions, use the ``just plug it in'' rule.
    \item (Example 10) Evaluate \(\lim_{x\to 4}x^2-7\)
    \item For limits of the form \(\frac{\text{nonzero}}{0}\), the limit is
          undefined.
    \item (Example 9) Show \(\lim_{x\to 2}\frac{1}{x-2}\) does not exist.
    \item For limits of the form \(\frac{0}{0}\), the limit is indeterminate:
          use canceling to determine its value.
    \item (Example 11) Evaluate \(\lim_{x\to 2}\frac{x^2-4}{x-2}\).
  \end{itemize}

\item\textit{
  HW: 25-44
}
\end{itemize}

\section*{3.3 The Derivative}

\begin{itemize}
\item Secant and Tangent Lines
  \begin{itemize}
    \item The slope of a secant line is given by \(\frac{\Delta y}{\Delta x}\).
    \item The slope of a tangent line is given by
          \(\lim_{\Delta x\to 0}\frac{\Delta y}{\Delta x}\).
    \item (Example) Find the slope of a few secant lines for \(y=x^2\) about the
          point \((2,4)\), use this to guess the slope of the tangent line
          at \((2,4)\), then calculate the tangent slope directly from the limit.
  \end{itemize}
\item Derivative
  \begin{itemize}
    \item The derivative \(f'(x)\) or \(\frac{d}{dx}[f(x)]\) of a function
          gives the slope of the tangent lines for each point on the graph
          \((x,f(x))\).
    \item \(f'(x)=\lim_{\Delta x\to 0}\frac{\Delta y}{\Delta x}=
            \lim_{\Delta x\to 0}\frac{f(x+\Delta x)-f(x)}{\Delta x}\)
    \item (Example) Show that the derivative of \(f(x)=x^2\)
          is \(f'(x)=2x\), then use this to find the slope of the tangent
          line at \((2,4)\).
    \item (Example 2) Prove that for \(y=6x-2x^3\), \(y'=\frac{dy}{dx}=6-6x^2\).
    \item (Example 4) Prove that for \(g(x)=x^2+\frac{1}{x+1}\),
          \(g'(x)=2x-\frac{1}{(x+1)^2}\).
  \end{itemize}

\item\textit{
  HW: 1-24
}
\end{itemize}

\section*{3.5 Derivatives of Polynomials}

\begin{itemize}
\item Derivatives of Constants and Identity
  \begin{itemize}
    \item \(\frac{d}{dt}[c]=0\)
    \item (Example 1) Calculate the \(\frac{dy}{dx}\) for \(y=-5\).
    \item (Example 3) Prove that if \(y=x\) then \(y'=1\).
  \end{itemize}
\item Derivatives of \(x^n\)
  \begin{itemize}
    \item (Example) Prove that if \(f(x)=x^5\) then \(f'(x)=5x^4\).
    \item \(\frac{d}{dx}[x^p]=px^{p-1}\)
    \item (Example 2) Find the derivative of \(y=x^3\).
    \item (Example 4) Find \(\frac{dv}{dr}\) where \(v=r^10\).
  \end{itemize}
\item Constant Multiple Rule
  \begin{itemize}
    \item \(\frac{d}{dx}[cf(x)]=cf'(x)\)
    \item (Example 5) Find the derivative of \(y=3x^2\).
  \end{itemize}
\item Sum/Difference Rule
  \begin{itemize}
    \item \(\frac{d}{dx}[f(x)+g(x)]=f'(x)+g'(x)\)
    \item (Example 7) Find the slope of a line tangent to the curve
          \(y=4x^7-x^4\) at the point \((1,3)\).
  \end{itemize}

\item\textit{
  HW: 1-18
}
\end{itemize}

\section*{3.4 The Derivative as an Instantaneous Rate of Change}

\begin{itemize}
\item Interpretation of \(\frac{du}{dv}\)
  \begin{itemize}
    \item The fraction \(\frac{\Delta u}{\Delta v}\) represents the
    change in a variable \(u\) as compared to the change in another variable \(v\).
    \item Therefore the expression
    \(\frac{du}{dv}=\lim_{\Delta v\to 0}\frac{\Delta u}{\Delta v}\) measures
    the instantaneous rate of change in \(u\) with respect to the rate of
    change in \(v\).
    \item In particular, if \(s\) is the position of an object and
    \(t\) is the time, then \(\frac{ds}{dt}\) is the instantaneous rate of
    change in position with respect to time, known as its velocity.
    \item (Example 3) Objects at sea level fall roughly \(16t^2\) feet after
    \(t\) seconds from release. Note that after \(4\) seconds, the object
    has fallen \(256\) feet. Use the following chart to approximate
    the instantaneous downward velocity of the object \(4\) seconds after
    release, then compute it exactly using a derivative.
    \[
      \begin{array}{c|c|c|c|c|c}
        t & 3 & 3.9 & 3.99 & 3.999 & 4 \\
        \Delta t \text{ from } 4 & 1 & 0.1 & 0.01 & 0.001 & 0 \\\hline
        s & 144 & 243.36 & 254.7216 & 255.872016 & 256 \\
        \Delta s \text{ from } 256 & 112 & 12.64 & 1.2784 & 0.127984 & 0 \\\hline\hline
        \frac{\Delta s}{\Delta t} & 112 & 126.4 & 127.84 & 127.984
      \end{array}
    \]
    \item (Example 5) A spherical balloon is being inflated. Find a formula
    for the instantaneous rate of change of volume with respect to its radius,
    then compute it when the radius is \(2\) meters. (Hint: \(V=\frac{4}{3}\pi r^3\).)
    \item (3.5 Example 8) Suppose the displacement of a piston is \(t^3-6t^2+8t\)
    centimeters after \(t\) seconds have elapsed. Find the position and
    velocity of the piston in one second intervals from \(t=0\) to \(t=4\).
  \end{itemize}

\item\textit{
  HW \textbf{in section 3.5}: 25-32, 38-42
}
\end{itemize}



\section*{Remaining Topics}

\begin{itemize}
  \item 3.6 Derivatives of Products and Quotients of Functions
  \item 3.7 The Derivative of a Power of a Function
  \item 3.8 Differentiation of Implicit Functions
  \item 3.9 Higher Derivatives
  \item 4.1 Tangents and Normals
  \item 4.4 Related Rates
  \item 4.5 Using Derivatives in Curve Sketching
  \item 4.6 More on Curve Sketching
  \item 4.7 Applied Maximum and Minimum Problems
  \item 4.8 Differentials and Linear Approximations
  \item 5.1 Antiderivatives
  \item 5.2 The Indefinite Integral
  \item 5.3 The Area Under a Curve
  \item 5.4 The Definite Integral
  \item 7.1 The Trigonometric Functions
  \item 7.2 Basic Trigonometric Relations
  \item 7.3 Derivatives of the Sine and Cosine Functions
  \item 7.4 Derivatives of the Other Trigonometric Functions
  \item 8.1 Exponential and Logarithmic Functions
  \item 8.2 Derivative of the Logarithmic Functions
  \item 8.3 Derivative of the Exponentials Function
  \item 9.1 The General Power Formula
  \item 9.2 Basic Logarithmic Form
  \item 9.3 Exponential Form
  \item 9.4 Basic Trigonometric Forms
\end{itemize}

\end{document}