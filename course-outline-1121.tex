\documentclass[11pt]{article}


\pdfpagewidth 8.5in
\pdfpageheight 11in

\setlength\topmargin{0in}
\setlength\headheight{0in}
\setlength\headsep{0.4in}
\setlength\textheight{8in}
\setlength\textwidth{6in}
\setlength\oddsidemargin{0in}
\setlength\evensidemargin{0in}
\setlength\parindent{0.25in}
\setlength\parskip{0.1in}

\usepackage{amssymb}
\usepackage{amsfonts}
\usepackage{amsmath}
\usepackage{mathtools}
\usepackage{amsthm}

\usepackage{fancyhdr}

\usepackage{enumerate}

      \theoremstyle{plain}
      \newtheorem{theorem}{Theorem}
      \newtheorem{lemma}[theorem]{Lemma}
      \newtheorem{corollary}[theorem]{Corollary}
      \newtheorem{proposition}[theorem]{Proposition}
      \newtheorem{conjecture}[theorem]{Conjecture}
      \newtheorem{question}[theorem]{Question}

      \theoremstyle{definition}
      \newtheorem{definition}[theorem]{Definition}
      \newtheorem{example}[theorem]{Example}
      \newtheorem{game}[theorem]{Game}

      \theoremstyle{remark}
      \newtheorem{remark}[theorem]{Remark}



\pagestyle{fancy}
\renewcommand{\headrulewidth}{0.5pt}
\renewcommand{\footrulewidth}{0pt}
\lfoot{\small \jobname{} -- Updated on \today}
\chead{\small Dr. Clontz -- Fall 2015}
\rfoot{\thepage}
\cfoot{}

\newcommand{\vect}[1]{\mathbf{#1}}
\newcommand{\veci}{\vect i}
\newcommand{\vecj}{\vect j}
\newcommand{\veck}{\vect k}
\newcommand{\<}{\langle}
\renewcommand{\>}{\rangle}
\newcommand{\Arctan}{\textrm{Arctan}}
\newcommand{\p}{\partial}
\newcommand{\mb}{\mathbb}
\newcommand{\sgn}{\textrm{sgn}}
\renewcommand{\div}{\textrm{div}\,}
\newcommand{\curl}{\textrm{curl}\,}
\newcommand{\scurl}{\textrm{scurl}\,}
\newcommand{\dvar}{\,d}

\renewcommand{\labelitemii}{\tiny$\blacksquare$}

\begin{document}

\noindent\textbf{
  MATH 1121 (Calculus for Engineering Technology) Course Outline
}

\section*{1.3 Rectangular Coordinates}

\begin{itemize}
\item Illustrate the following concepts:
  \begin{itemize}
    \item rectangular coordinate system,
    \item \(x\)-axis,
    \item \(y\)-axis,
    \item origin,
    \item quadrants,
    \item coordinates
  \end{itemize}

\item Examples:
  \begin{itemize}
    \item (Example 1) Plot \(A=(2,1)\) and \(B=(-4,-3)\).
    \item (Example 3) Three verticies of a rectangle are
          \(A=(-3,-2)\), \(B=(4,-2)\), \(C=(4,1)\). What is the
          fourth vertex?
  \end{itemize}

\item\textit{
  HW: 1-9, 15-16, 21-24
}
\end{itemize}

\section*{2.1 Some Basic Definitions}

\begin{itemize}
\item Distance Formula
  \begin{itemize}
    \item \(d=\sqrt{(x_2-x_1)^2+(y_2-y_1)^2}\)
    \item (Example 2) Find the distance between \((3,-1)\) and \((-2,-5)\).
  \end{itemize}
\item Slope Formula
  \begin{itemize}
    \item \(m=\frac{y_2-y_1}{x_2-x_1}\)
    \item (Example 3) Find the slope of the line joining
          \((3,-5)\), \((-2,-6)\).
    \item (Example 4) Find the slope of the line joining
          \((3,4)\), \((4,-6)\).
    \item \(m=\tan\theta\)
    \item (Example) Find the slope of the line with inclination \(120^\circ\).
  \end{itemize}
\item Identify parallel/perpendicular lines by slopes.
  \begin{itemize}
    \item Parallel: \(m_1=m_2\)
    \item Perpendicular: \(m_1=-\frac{1}{m_2}\)
    \item (Example 7) Prove that the triangle with vertices
          \(A=(-5,3)\), \(B=(6,0)\), and \(C=(5,5)\) is a right triangle.
  \end{itemize}
\item\textit{
  HW: 1-20, 29-36
}
\end{itemize}

\section*{2.2 The Straight Line}

\begin{itemize}
\item Point-slope form
  \begin{itemize}
    \item \(y-y_1=m(x-x_1)\)
    \item (Example 2) Find the equation of the line passing through
          \((2,-1)\) and \((6,2)\).
  \end{itemize}
\item Slope-intercept form
  \begin{itemize}
    \item \(y=mx+b\)
    \item (Example 4) Find the slope and \(y\)-intercept of the straight line
          with equation \(2y+4x-5=0\).
  \end{itemize}
\item\textit{
  HW: 1-21, 33-40
}
\end{itemize}

\section*{2.3 The Circle}

\begin{itemize}
  \item Definition
    \begin{itemize}
      \item A circle is a collection of points equidistant from its center.
    \end{itemize}
  \item Standard form
    \begin{itemize}
      \item \((x-h)^2+(y-k)^2=r^2\)
      \item (Example 1) Sketch \((x-1)^2+(y+2)^2=16\).
      \item (Example 2) Find an equation for the circle with center \((2,1)\)
            which passes through \((4,8)\).
    \end{itemize}
  \item General form
    \begin{itemize}
      \item \(x^2+y^2+Dx+Ey+F=0\)
      \item (Example 4) Find the center and radius of the circle
            \(x^2+y^2-6x+8y-24=0\).
      \item (Example) Find two functions whose graphs represent the circle
            with the previous equation.
    \end{itemize}
  \item\textit{
    HW: 1-32, 37-38
  }
\end{itemize}

\section*{2.4 The Parabola}

\begin{itemize}
  \item Definition
    \begin{itemize}
      \item A parabola is a colleciton of points equidistant from a focus point
        and a directrix line.
      \item The vertex of a parabola is the point closest to the focus and
        directix.
      \item (Example 6) Find an equation for the parabola with focus \((2,3)\)
        and directrix \((y=-1)\)
    \end{itemize}
  \item Standard forms with vertex at origin and horizontal/vertical directrix
  %TODO simplify to two equations
    \begin{itemize}
      \item \(y^2=4px\) with directrix at \(x=-p\) and focus at \((p,0)\)
      \item \(y^2=-4px\) with directrix at \(x=p\) and focus at \((-p,0)\)
      \item \(x^2=4py\) with directrix at \(y=-p\) and focus at \((0,p)\)
      \item \(x^2=-4py\) with directrix at \(y=p\) and focus at \((0,-p)\)
      \item (Example 2) Find an equation for the parabola with focus
            \((-2,0)\) and directrix \((x=2)\).
      \item (Example 4) Find the focus and directrix of the parabola with
            equation \(2x^2=-9y\).
    \end{itemize}
  \item\textit{
    HW: 1-22, 25-28
  }
\end{itemize}

\section*{2.5 The Ellipse}

\begin{itemize}
  \item Definition
    \begin{itemize}
      \item An ellipse is a collection of points where the sum of distances
            from two fixed points (called foci) is kept constant.
      \item The two points furthest/closest apart from each other on an ellipse
            are the endpoints of the major/minor axis.
      \item The sum of distances
            between each point and the foci is the same as the length of
            the major axis. The major axis passes through both foci.
    \end{itemize}
  \item Standard form with center at the origin
    \begin{itemize}
      \item \(\frac{x^2}{a^2}+\frac{y^2}{b^2}=1\), with foci given by
            \((c,0),(-c,0)\), where
            \(a^2-b^2=c^2\)
      \item \(\frac{y^2}{a^2}+\frac{x^2}{b^2}=1\), with foci given by
            \((0,c),(0,-c)\), where
            \(a^2-b^2=c^2\)
      \item (Example 3) Sketch the ellipse with equation \(4x^2+16y^2=64\),
            and compute the locations of its foci.
      \item (Example 5) Find the equation of the ellipse centered at the origin
            with an end of its minor axis at \((2,0)\) and containing the point
            \((-1,\sqrt 6)\).
    \end{itemize}
  \item\textit{
    HW: 1-26
  }
\end{itemize}

\section*{2.6 The Hyperbola}

\begin{itemize}
  \item Definition
    \begin{itemize}
      \item A hyperbola is a collection of points where the difference of distances
            from two fixed points (called foci) is kept constant.
      \item Hyperbolas are split into two curves. The two closest points
            on opposite curves are called vertices and give the transverse axis.
    \end{itemize}
  \item Standard form with center at the origin
    \begin{itemize}
      \item \(\frac{x^2}{a^2}-\frac{y^2}{b^2}=1\), with foci given by
            \((c,0),(-c,0)\) and asymptotes \(y=\pm\frac{bx}{a}\),
            where \(a^2+b^2=c^2\)
      \item  \(\frac{y^2}{a^2}-\frac{x^2}{b^2}=1\), with foci given by
            \((0,c),(0,-c)\) and asymptotes \(x=\pm\frac{by}{a}\),
            where \(a^2+b^2=c^2\)
      \item (Example 2) Sketch \(\frac{y^2}{4}-\frac{x^2}{16}=1\), labeling
            its vertices, asymptotes, and foci.
      \item (Example 3) Sketch \(4x^2-9y^2=36\), labeling
            its vertices, asymptotes, and foci.
    \end{itemize}
  \item Hypberbola with coordinate axis asymptotes
    % TODO: change c^2 to a^2
    \begin{itemize}
      \item \(xy=c^2\), with vertices given by \((c,c)\) and \((-c,-c)\)
      \item \(xy=-c^2\), with vertices given by \((c,-c)\) and \((-c,c)\)
      \item (Example 5) Sketch \(xy=4\).
    \end{itemize}
  \item\textit{
    HW: 1-14, 17-24
  }
\end{itemize}

\section*{2.7 Translation of Axes}

\begin{itemize}
  \item Vertical/horizontal translation:
    \begin{itemize}
      \item Shift right \(h\): replace \(x\) with \(x-h\).
      \item Shift up \(k\): replace \(y\) with \(y-k\).
      \item (Example 1) Give an equation of the parabola with vertex \((2,4)\)
            and focus \((4,4)\).
      \item (Example 2) Sketch the curve with equation
            \(\frac{(x-3)^2}{25}+\frac{(y+2)^2}{9}=1\).
    \end{itemize}
  \item\textit{
    HW: 1-36
  }
\end{itemize}

\section*{1.2 Algebraic Functions}

\begin{itemize}
\item Definition of a function \(y=f(x)\).
\item Types of functions
  \begin{itemize}
    \item Polynomials \(P(x)=a_0+a_1x+\dots+a_nx^n\)
    \item Rational functions \(R(x)=\frac{P(x)}{Q(x)}\) for polynomials \(P,Q\)
    \item (Example 1) Voltage equals current multiplied by resistance. If the
          voltage at time \(t\) is given by \(E(t)=2t^2=y+5\) and the
          resistance at time \(t\) is given by \(R(t)=3t+20\), then find
          a function \(I(t)\) which measures the current at time \(t\).
          Identify it as a polynomial and/or rational function.
  \end{itemize}
\item Combinations of functions
  \begin{itemize}
    \item Addition/Subtraction/Multiplication/Division
    \item Compositions \(f\circ g\) and \(g\circ f\)
    \item (Example 2) Express \(f+g\), \(f\circ g\), and
          \(g\circ f\) for the functions
          given by \(f(x)=2x^2-3\) and \(g(x)=\sqrt{x+2}\).
  \end{itemize}
\item Domain/Range
  \begin{itemize}
    \item The domain of a function is all real numbers which may be plugged
          into it without causing division by zero, even roots of negatives,
          or any other undefined operations.
    \item The range of a function is all real numbers which may possibly
          be attained by the function.
    \item (Example 5) Find the domain and range of \(f(x)=x^2+2\) and
          \(g(t)=\frac{1}{t+2}\).
    \item (Example 7) Find the domain of \(f(x)=16\sqrt x+\frac{1}{x}\).
  \end{itemize}
\item Piecewise functions
  \begin{itemize}
    \item Piecewise functions are defined differently for different parts of
          their domains.
    \item (Example 9) Find the domain for
          \[ f(t) =
          \begin{cases}
            8-2t & 0\leq t\leq 4 \\
            0 & t>4
          \end{cases}
          \]
          and compute \(f(3),f(6),f(-1)\) if possible.
  \end{itemize}
\item Exponent laws
  \begin{itemize}
    \item \(a^ma^n=a^{m+n}\)
    \item \(\frac{a^m}{a^n}=a^{m-n}\)
    \item \((a^m)^n=a^{mn}\)
    \item \((ab)^m=a^mb^m\)
    \item \((\frac{a}{b})^m=\frac{a^m}{b^m}\)
    \item \(a^0=1\)
    \item \(a^{-n}=\frac{1}{a^n}\)
    \item \(a^{1/n}=\sqrt[n]{a}\)
    \item Note \(\sqrt{a^2}=|a|\) but \(\sqrt[3]{a^3}=a\)
    \item (Example 4) Simplify
          \[
            f(x)
              =
            \frac{
              (3x^2-1)^{1/3}(2x)-(2x^3)(3x^2-1)^{-2/3}
            }{
              (3x^2-1)^{2/3}
            }
          \]
  \end{itemize}
  \item\textit{
    HW: 1-18, 21-34
  }
\end{itemize}

\section*{1.4 The Graph of a Function}

\begin{itemize}
\item Definition
  \begin{itemize}
    \item The graph of a function is the collection of all ordered pairs
      \((x,y)\) such that \(y=f(x)\)
    \item Graphing Method 1: using Chapter 2
    \item Graphing Method 2: using \(xy\) chart
    \item Vertical line test: the graph of any function hits every vertical
      line at most once
  \end{itemize}
\item Examples
  \begin{itemize}
  \item (Example 1) Graph \(f(x)=3x-5\).
  \item (Example 3) Graph \(f(x)=1+\frac{1}{x}\).
  \item (Example 4) Graph \(f(x)=\sqrt{x+1}\).
  \item (Example 6) Graph
    \[
      f(x)=
        \begin{cases}
          2x+1 & x\leq 1 \\
          6-x^2 & x>1
        \end{cases}
    \]
  \end{itemize}

\item\textit{
  HW: 1-12, 37-40
}
\end{itemize}

\section*{3.1 Limits}

\begin{itemize}
\item Limits
  \begin{itemize}
    \item \(\lim_{x\to a}f(x)=L\) means that the value of \(f(x)\) approaches
          \(L\) as the value of \(x\) approaches \(a\) in the domain of \(f\).
    \item (Example) Given \(f(x)=x^2\), we may write the following chart of
          values

      \begin{tabular}{c|c}
        \(x\) & \(f(x)\) \\\hline
        \(1.9\) & \(3.61\) \\
        \(1.99\) & \(3.9601\) \\
        \(1.999\) & \(3.996001\) \\
        \(2.001\) & \(4.004001\) \\
        \(2.01\) & \(4.0401\) \\
        \(2.1\) & \(4.41\)
      \end{tabular}

          to infer that \(\lim_{x\to 2} f(x)=4\).
    \item (Example) Given
      \[
        g(x)
          =
        \left\{
        \begin{matrix}
          x^2 & x\not= 2 \\
          -5 & x=2
        \end{matrix}
        \right.
      \]
          we have the same chart of values as before, so we assume
          \(\lim_{x\to 2} g(x)=4\).
    \item (Example) Since \(h(x)=\frac{x^3-2x^2}{x-2}\) equals \(x^2\) for
      all values of \(x\) except \(2\), we have the same chart of values
      as before, and we assume \(\lim_{x\to 2} h(x)=4\).
    \item (Example) By graphing \(y=f(x)\), \(y=g(x)\), and \(y=h(x)\),
      we can see that the points on the graph approach the point \((2,4)\)
      in all three cases.
  \end{itemize}
\item Continuity
  \begin{itemize}
    \item A continuous function satsifies the equality \(f(a)=\lim_{x\to a}f(x)\)
    for all numbers \(a\) in its domain. (The ``just plug it in'' rule.)
    \item Intuitively: the graph of the function can be drawn without lifting
    your pencil on the intervals where it is defined
    \item FACT: \(f(x)=x\) is continuous, and any combination of continuous
    functions using \(+\), \(-\), \(\times\), \(/\), \(\circ,\) or powers is
    continuous (where it is defined).
    \item (Example 3) \(f(x)=\frac{1}{x-2}\) is continuous for its entire
    domain, but undefined at its asymptote \(x=2\).
    \item (Example 5) By graphing
      \[
        \begin{matrix}
        f(x)
          =
        \left\{
        \begin{matrix}
          x+2 & x < 1 \\
          -\frac{x}{2}+5 & x\geq 1
        \end{matrix}
        \right.
          &
        g(x)
          =
        \left\{
        \begin{matrix}
          2x-1 & x\leq 2 \\
          -x+5 & x>2
        \end{matrix}
        \right.
        \end{matrix}
      \]
    we see that \(f\) is continuous except for when \(x=1\), and
    \(g\) is continuous everywhere.
  \end{itemize}
\item Limits to \(\pm\infty\)
  \begin{itemize}
    \item \(\lim_{x\to \infty}f(x)=L\) means that the value of \(f(x)\) approaches
          \(L\) as the value of \(x\) attains arbitrarily large postive values.
    \item \(\lim_{x\to -\infty}f(x)=L\) means that the value of \(f(x)\) approaches
          \(L\) as the value of \(x\) attains arbitrarily large negative values.
    \item (Example) Use a chart of values to infer that
          \(\lim_{x\to\pm\infty}\frac{1}{x}=0\).
    \item (Example 14) Use a chart of values and algebraic manipulation to
          show that \(\lim_{x\to\pm\infty}\frac{x^2+1}{2x^2+3}=\frac{1}{2}\).
  \end{itemize}
\item Evaluating limits analytically
  \begin{itemize}
    \item For continuous functions, use the ``just plug it in'' rule.
    \item (Example 10) Evaluate \(\lim_{x\to 4}x^2-7\)
    \item For limits of the form \(\frac{\text{nonzero}}{0}\), the limit is
          undefined.
    \item (Example 9) Show \(\lim_{x\to 2}\frac{1}{x-2}\) does not exist.
    \item For limits of the form \(\frac{0}{0}\), the limit is indeterminate:
          use canceling to determine its value.
    \item (Example 11) Evaluate \(\lim_{x\to 2}\frac{x^2-4}{x-2}\).
  \end{itemize}

\item\textit{
  HW: 25-44
}
\end{itemize}

\section*{3.3 The Derivative}

\begin{itemize}
\item Secant and Tangent Lines
  \begin{itemize}
    \item The slope of a secant line is given by \(\frac{\Delta y}{\Delta x}\).
    \item The slope of a tangent line is given by
          \(\lim_{\Delta x\to 0}\frac{\Delta y}{\Delta x}\).
    \item (Example) Find the slope of a few secant lines for \(y=x^2\) about the
          point \((2,4)\), use this to guess the slope of the tangent line
          at \((2,4)\), then calculate the tangent slope directly from the limit.
  \end{itemize}
\item Derivative
  \begin{itemize}
    \item The derivative \(f'(x)\) or \(\frac{d}{dx}[f(x)]\) of a function
          gives the slope of the tangent lines for each point on the graph
          \((x,f(x))\).
    \item \(f'(x)=\lim_{\Delta x\to 0}\frac{\Delta y}{\Delta x}=
            \lim_{\Delta x\to 0}\frac{f(x+\Delta x)-f(x)}{\Delta x}\)
    \item (Example) Show that the derivative of \(f(x)=x^2\)
          is \(f'(x)=2x\), then use this to find the slope of the tangent
          line at \((2,4)\).
    \item (Example 2) Prove that for \(y=6x-2x^3\), \(y'=\frac{dy}{dx}=6-6x^2\).
    \item (Example 4) Prove that for \(g(x)=x^2+\frac{1}{x+1}\),
          \(g'(x)=2x-\frac{1}{(x+1)^2}\).
  \end{itemize}

\item\textit{
  HW: 1-24
}
\end{itemize}

\section*{3.5 Derivatives of Polynomials}

\begin{itemize}
\item Derivatives of Constants and Identity
  \begin{itemize}
    \item \(\frac{d}{dx}[c]=0\)
    \item (Example 1) Calculate the \(\frac{dy}{dx}\) for \(y=-5\).
    \item \(\frac{d}{dx}[x]=1\)
    \item (Example 3) Prove that if \(y=x\) then \(y'=1\).
  \end{itemize}
\item Derivatives of \(x^n\)
  \begin{itemize}
    \item (Example) Prove that if \(f(x)=x^5\) then \(f'(x)=5x^4\).
    \item \(\frac{d}{dx}[x^p]=px^{p-1}\)
    \item (Example 2) Find the derivative of \(y=x^3\).
    \item (Example 4) Find \(\frac{dv}{dr}\) where \(v=r^{10}\).
  \end{itemize}
\item Constant Multiple Rule
  \begin{itemize}
    \item \(\frac{d}{dx}[cf(x)]=cf'(x)\)
    \item (Example 5) Find the derivative of \(y=3x^2\).
  \end{itemize}
\item Sum/Difference Rule
  \begin{itemize}
    \item \(\frac{d}{dx}[f(x)+g(x)]=f'(x)+g'(x)\)
    \item (Example 7) Find the slope of a line tangent to the curve
          \(y=4x^7-x^4\) at the point \((1,3)\).
  \end{itemize}

\item\textit{
  HW: 1-18
}
\end{itemize}

\section*{3.4 The Derivative as an Instantaneous Rate of Change}

\begin{itemize}
\item Interpretation of \(\frac{du}{dv}\)
  \begin{itemize}
    \item The fraction \(\frac{\Delta u}{\Delta v}\) represents the
    change in a variable \(u\) as compared to the change in another variable \(v\).
    \item Therefore the expression
    \(\frac{du}{dv}=\lim_{\Delta v\to 0}\frac{\Delta u}{\Delta v}\) measures
    the instantaneous rate of change in \(u\) with respect to the rate of
    change in \(v\).
    \item In particular, if \(s\) is the position of an object and
    \(t\) is the time, then \(\frac{ds}{dt}\) is the instantaneous rate of
    change in position with respect to time, known as its velocity.
    \item (Example 3) Objects at sea level fall roughly \(16t^2\) feet after
    \(t\) seconds from release. Note that after \(4\) seconds, the object
    has fallen \(256\) feet. Use the following chart to approximate
    the instantaneous downward velocity of the object \(4\) seconds after
    release, then compute it exactly using a derivative.
    \[
      \begin{array}{c|c|c|c|c|c}
        t & 3 & 3.9 & 3.99 & 3.999 & 4 \\
        \Delta t \text{ from } 4 & 1 & 0.1 & 0.01 & 0.001 & 0 \\\hline
        s & 144 & 243.36 & 254.7216 & 255.872016 & 256 \\
        \Delta s \text{ from } 256 & 112 & 12.64 & 1.2784 & 0.127984 & 0 \\\hline\hline
        \frac{\Delta s}{\Delta t} & 112 & 126.4 & 127.84 & 127.984 & \text{\tiny(DNE)}
      \end{array}
    \]
    \item (Example 5) A spherical balloon is being inflated. Find a formula
    for the instantaneous rate of change of volume with respect to its radius,
    then compute it when the radius is \(2\) meters. (Hint: \(V=\frac{4}{3}\pi r^3\).)
    \item (3.5 Example 8) Suppose the displacement of a piston is \(t^3-6t^2+8t\)
    centimeters after \(t\) seconds have elapsed. Find the position and
    velocity of the piston in one second intervals from \(t=0\) to \(t=4\).
  \end{itemize}

\item\textit{
  HW \textbf{in section 3.5}: 25-32, 38-42
}
\end{itemize}

\section*{3.6 Derivatives of Products and Quotients of Functions}

\begin{itemize}
\item Product Rule
  \begin{itemize}
    \item (Example 2) Find the derivative of the function
          \(p(x)=(x^2+2)(3-2x)\).
    \item Product Rule:
          \(\frac{d}{dx}[f(x)g(x)]=g(x)f'(x)+f(x)g'(x)\)
    \item (Example 2 again) Verify the product rule.
    \item (Example 3) Find \(\frac{dy}{dx}\) where \(y=(3-x-2x^2)(x^4-x)\).
  \end{itemize}
\item Quotient Rule
  \begin{itemize}
    \item (Example) Find the derivative of the function
          \(q(x)=\frac{x^2+1}{x}\).
    \item Quotient Rule:
          \(\frac{d}{dx}[\frac{f(x)}{g(x)}]=\frac{g(x)f'(x)-f(x)g'(x)}{[g(x)]^2}\)
    \item (Same Example) Verify the quotient rule.
    \item (Example 4) Find the derivative of \(h(x)=\frac{3-2x}{x^2+2}\).
    \item (Example 5) The stress \(S\) on a hollow tube with tension \(T\),
          outer diameter \(D\), and inner diameter \(d\) is given by the
          equation \(S=\frac{16DT}{\pi(D^4-d^4)}\). Assume this tube has
          constant tension \(T=10\) and constant inner diameter \(d=1\).
          Find the rate of change
          stress increases with respect to an increasing outer diameter
          when \(D=2\).
  \end{itemize}
\item\textit{
  HW: 1-28, 39-42
}
\end{itemize}

\section*{3.7 The Derivative of a Power of a Function}

\begin{itemize}
\item Chain Rule for Power Functions
  \begin{itemize}
    \item (Example 1) Find the derivative of the function
          \(y=(3-2x)^3\).
    \item Chain Rule for Powers:
          \(\frac{d}{dx}[(f(x))^p]=p(f(x))^{p-1}f'(x)\)
    \item (Example 1 again) Verify the chain rule.
    \item (Example 5) Find the derivative of \(y=6\sqrt[3]{x^2}\).
    \item (Example 4) Find the derivative of \(y=\sqrt{x^2+1}\).
    \item (Example 8) Evaluate the derivative of \(y=\frac{x}{\sqrt{1-4x}}\)
          when \(x=-2\).
  \end{itemize}
\item\textit{
  HW: 1-24, 29-30, 35-38
}
\end{itemize}

\section*{3.8 Differentiation of Implicit Functions}

\begin{itemize}
\item Implicit Functions
  \begin{itemize}
    \item The expression \(y=f(x)\) defines an explicit function.
    \item An equation with variables \(x,y\) may define \(y\) as an
          implicit function of \(x\).
    \item (Example 1) Manipulate the equation \(3x+4y=5\) which defines
          \(y\) as an implicit function of \(x\) so that \(y\) is defined
          as an explicit function of \(x\).
    \item (Example) Give an explicit function which describes the part of
          the hyperbola centered at the origin
          with focus \((0,5)\) and vertex \((0,3)\) passing through
          the point \((-16/3,5)\).
  \end{itemize}
\item Implicit Differentiation
  \begin{itemize}
    \item (Example) Find the slope of the line tangent to the hyperbola
          from the previous example at the point \((-16/3,5)\).
    \item Implicit functions may be differentated directly by using the
          chain rule: differentiate \(y\) terms as you would \(x\), but
          tack on a \(\frac{dy}{dx}\) term each time you do.
    \item (Example) Find the slope of the line described in Example 1 using both
          implicit and explicit differentation.
    \item (Example) Solve the hyperbola problem using implicit differentiation.
    \item (Example 3) Find \(\frac{dy}{dx}\) in terms of \(x,y\) where
          \(3y^4+xy^2=6-2x^3\).
    \item (Example 5) Find the slope of a line tangent to the graph of
          \(2y^3+xy+1=0\) at the point \((-3,1)\).
  \end{itemize}
\item\textit{
  HW: 1-25, 28-30
}
\end{itemize}

\section*{3.9 Higher Derivatives}

\begin{itemize}
\item Higher Derivatives
  \begin{itemize}
    \item The derivative of a derivative is its second derivative. The
          derivative of a second derivative is its third derivative, etc.
    \item (Example 1) Find all higher derivatives of \(y=5x^3-2x\).
    \item (Example 3) Find the second derivative of \(y=\frac{2}{1-x}\) when
          \(x=-2\).
    \item (Example 4) Find \(y''\) for the implicit function defined by
          \(2x^2+3y^2=6\) in terms of \(x,y\).
  \end{itemize}
\item Acceleration
  \begin{itemize}
    \item If \(s\) is position defined in terms of \(t\) (time), then
          \(s'=\frac{ds}{dt}\) is velocity and \(s''=\frac{d^2s}{dt^2}\)
          is acceleration.
    \item (Example) The height of an object launched upward from the ground
          with an initial velocity of \(v_0\) m/s is roughly
          \(s=-4.9t^2+v_0t\) meters after \(t\) seconds.
          Find the velocity and acceleration of this object after \(1\) second
          given its initial velocity \(v_0=10\) meters per second.
  \end{itemize}
\item\textit{
  HW: 1-34, 37-38
}
\end{itemize}

\section*{4.4 Related Rates}

\begin{itemize}
\item Related Rates as Implicit Differentation
  \begin{itemize}
    \item If the variables in an equation are functions of time, then
          we may use implicit differentiation to compare their rates of
          change with respect to time.
    \item (Example 1) The voltage \(E\) of a certain thermocouple may
          be measured as \(E=2.8T+0.006 T^2\) where \(T\) is its temperature
          in Celcius. If the temperature of the thermocouple is increasing
          at a rate of \(1^\circ\) C/min, then how fast is the voltage
          increasing when the temperature is \(100^\circ\) C?
    \item (Example 3) A spherical balloon is being filled at a rate of
          \(2\) cubic feet per minute. How fast is its radius growing
          when the radius is \(3\) feet long?
    \item (Example 5) Two ships leave port at noon. Ship \(A\) travels
          west at \(12\) km/h, and ship \(B\) travels south at \(16\) km/h.
          Show that their relative speed is \(20\) km/h at 2pm.
          (Note: it's actually always \(20\) km/h, but it's easier to solve
          at a specific time.)
    \item (Example) A \(13\) foot ladder is slipping down a vertical wall.
          The top of the vertical ladder is \(13-2t^2\) feet high after \(t\)
          seconds of slipping. How fast is the bottom of the ladder extending
          from the base of the wall after \(2\) seconds?
    \item (Example) A football is spiralling directly towards the ground
          at a rate of \(2\) yards per second
          from a height of \(10\) yards. A \(50\) yard tall lightpost shines on
          the ball. The base of the lightpost is currently
          \(100\) yards away from
          the ball's shadow (but of course, that will change with
          time). How fast is the ball's shadow moving along the
          ground?
  \end{itemize}
\item\textit{
  HW: 1-24
}
\end{itemize}

% \section*{4.5 Using Derivatives in Curve Sketching}

% \begin{itemize}
% \item Graphical interpretations of derivatives
%   \begin{itemize}
%     \item Table of relationships between \(f'\) and \(f''\) and the graph
%           of \(f\): \\
%       \begin{tabular}{c|c|c|c|c}
%         \(f'\) & \(+\) & \(+\) & \(-\) & \(-\) \\
%         Graph of \(f\) & increasing & increasing & decreasing & decreasing \\
%         \(f''\) & \(+\) & \(-\) & \(+\) & \(-\) \\
%         Graph of \(f\) & concave up & concave down & concave up & concave down
%       \end{tabular}
%     \item Graphing polynomials with derivatives:
%       \begin{itemize}
%         \item Find your critical values where \(f'(x)=0\).
%         \item Find your inflection candidates where \(f''(x)=0\).
%         \item Label these on a number line.
%         \item Choose numbers between the labeled values and test them in
%               both \(f'\) and \(f''\).
%         \item Plot the critcal values and inflection candidates on the
%               coordinate plane.
%         \item Connect the dots using the increasing/decreasing and concavity
%               information from \(f'\), \(f''\).
%       \end{itemize}
%     \item (Example 4) Sketch the graph of \(y=6x-x^2\).
%     \item (Example 5) Sketch the graph of \(y=2x^3+3x^2-12x\).
%     \item A maximum point occurs when a graph changes from increasing to
%           decreasing. A minimum point occurs when a graph changes from
%           decreasing to increasing.
%     \item An inflection point occurs when the concavity of a graph
%           changes.
%     \item (Example) Label the maximum points, minimum points, and inflection
%           points on the graph of \(y=2x^3+3x^2-12x\).
%   \end{itemize}
% \item\textit{
%   HW: 13-28
% }
% \end{itemize}

\section*{4.7 Applied Maximum and Minimum Problems}

\begin{itemize}
\item Optimizing functions
  \begin{itemize}
    \item The optimal (max or min) value of a function must occur either
          at a critical value or at an endpoint of its domain.
    % \item Solving optimization problems:
    %   \begin{itemize}
    %     \item Draw a picture, labeling with variables.
    %     \item Model the problem with an equation.
    %     \item Solve for the variable \(Q\) you wish to optimize.
    %     \item Reduce all other variables to a single variable \(x\).
    %     \item Determine the domain of \(x\), particularly the least and
    %           greatest values of \(x\) allowed.
    %     \item Find all critical values of \(Q\) as a function of \(x\);
    %           that is, find all values of \(x\) where \(\frac{dQ}{dx}=0\).
    %     \item Test all critical values and domain endpoints to determine
    %           the value of \(x\) which gives the optimal value of \(Q\).
    %   \end{itemize}
    \item (Example 3) Maximize the area of a rectangular corral built with
          \(1600\) feet of fencing.
    \item (Example 2) Find the number which exceeds its square by the greatest
          amount.
    \item (Example 5) Find the point on the parabola \(y=x^2\) closest to
          the point \((6,3)\).
    \item (Example 6) A company sells \(1000\) widgets a month when sold at
          a price of \(\$5\) per widget. For every \(\$0.01\) reduction in
          price, the company will sell \(10\) more units a month. What price
          should the company set for each widget in order to maximize sales
          in dollars?
  \end{itemize}
\item\textit{
  HW: 1-18,23-27
}
\end{itemize}

\section*{4.8 Differentials and Linear Approximations}

\begin{itemize}
\item Linear approximations
  \begin{itemize}
    \item The line tangent to a function approximates the function near that
          point.
    \item \(L_a(x)=f(a)+f'(a)(x-a)\) approximates the function \(f(x)\) near \(x=a\).
    \item (Example 8) Approximate the value of \(\sqrt{9.06}\) using the
          linear approximation \(L_4(x)\) of \(f(x)=\sqrt{2x+1}\) at \(x=4\).
          (A calculator gives \(3.00998\).)
    \item (Example) Find the linear function \(L_0(x)\) which approximates
          \(g(x)=3x^3-4x+7\) near \(x=0\), then approximate \(g(0.1)\).
          (The exact value is \(6.603\).)
    \item (Example) Use \(f(x)=\frac{1}{\sqrt{3+x^2}}\) to show
          that \(\frac{1}{\sqrt{4.21}}\approx 0.4875\).
          (A calculator gives \(0.48737\).)
  \end{itemize}
\item\textit{
  HW: 21-24, 32-34
}
\end{itemize}

\section*{5.2 The Indefinite Integral}

\begin{itemize}
\item Antiderivatives
  \begin{itemize}
    \item If \(F'(x)=f(x)\), then \(F(x)\) is an antiderivative of \(f(x)\).
    \item (Example) Find a few antiderivatives of \(f(x)=5x^4\).
  \end{itemize}
\item Indefinite Integral
  \begin{itemize}
    \item \(\int f(x)\dvar x\) represents all antiderivatives of \(f(x)\).
    \item \(\int f(x)\dvar x=F(x)+C\).
    \item (Example 2) Find \(\int 5x^4 \dvar x\).
  \end{itemize}
\item Reverse Power Rule
  \begin{itemize}
    \item \(\int x^p\dvar x = \frac{1}{p+1}x^{p+1}\)
    \item (Example) Find \(\int x^7\dvar x\).
  \end{itemize}
\item Constant Multiple and Sum/Difference Rules
  \begin{itemize}
    \item \(\int cf(x)\dvar x=cF(x)+C\)
    \item (Example 3) Find \(\int 6x\dvar x\).
    \item \(\int f(x)\pm g(x)\dvar x=F(x)\pm G(x)+C\)
    \item (Example 4) Find \(\int 5x^3-6x^2+1\dvar x\).
    \item (Example 5) Find \(\int \sqrt r - \frac{1}{r^3}\dvar r\).
  \end{itemize}
\item Substitution
  \begin{itemize}
    \item Complicated integrals may be evaluated by substituting \(u=u(x)\)
          and \(du=u'(x)dx\).
    \item (Example 6) Find \(\int(x^2+1)^3(2x)\dvar x\).
    \item (Example 7) Find \(\int x^2\sqrt{x^3+2}\dvar x\).
  \end{itemize}
\item Finding the constant of integration
  \begin{itemize}
    \item While \(\int f(x)\dvar x=F(x)+C\) represents all antiderivatives
          of \(f(x)\), we sometimes need to solve for \(C\) to get a specific
          antiderivative.
    \item (Example 8) Find an equation for \(y\) in terms of \(x\) where
          \(\frac{dy}{dx}=3x-1\) and its graph passes through \((1,4)\).
    \item (Example 9) Find an equation for the displacement of an object
          with velocity given by \(v=t\sqrt{9-t^2}\) assuming \(s=0\)
          when \(t=0\).
  \end{itemize}
\item\textit{
  HW: 1-18, 21-27, 33-36
}
\end{itemize}

\section*{5.3 The Area Under a Curve}

\begin{itemize}
\item Derivation
  \begin{itemize}
    \item Let \(\Delta A\) be the area under an increasing curve \(y=f(x)\) from
          \(x\) to \(x+\Delta\).
    \item Then \(f(x)\Delta x\leq \Delta A\leq f(x+\Delta x)\Delta x\)
          and \(f(x)\leq\frac{\Delta A}{\Delta x}\leq f(x+\Delta x)\).
    \item As \(\Delta x\to 0\), we get \(\frac{dA}{dx}=f(x)\).
    \item Thus \(A(x)=\int f(x)\dvar x\)
          can be used to measure the area under a curve between points.
  \end{itemize}
\item Formula
  \begin{itemize}
    \item Let \(F(x)\) be any antiderivative of \(f(x)\) (usually assuming
          \(C=0\)). Then the definite integral
          \(\int_a^b f(x)\dvar x = \left[F(x)\right]_a^b=F(b)-F(a)\) measures
          the area under the curve \(y=f(x)\) from \(x=a\) to \(x=b\).
    \item (Example 2) Find the area under the straight line \(y=2x\)
          between \(x=0\) and \(x=4\).
    \item (Example 4) Find the area under the curve \(y=x^3\) for
          \(1\leq x\leq 2\).
    \item (Example 5) Find the area in the first quadrant between the
          coordinate axes and \(y=4-x^2\).
  \end{itemize}
\item\textit{
  HW: 11-19
}
\end{itemize}



\section*{Remaining Topics}

\begin{itemize}
  \item 7.1 The Trigonometric Functions
  \item 7.2 Basic Trigonometric Relations
  \item 7.3 Derivatives of the Sine and Cosine Functions
  \item 7.4 Derivatives of the Other Trigonometric Functions
  \item 8.1 Exponential and Logarithmic Functions
  \item 8.2 Derivative of the Logarithmic Functions
  \item 8.3 Derivative of the Exponentials Function
  \item 9.1 The General Power Formula
  \item 9.2 Basic Logarithmic Form
  \item 9.3 Exponential Form
  \item 9.4 Basic Trigonometric Forms
\end{itemize}

\end{document}