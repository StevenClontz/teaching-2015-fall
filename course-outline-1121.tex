\documentclass[11pt]{article}


\pdfpagewidth 8.5in
\pdfpageheight 11in

\setlength\topmargin{0in}
\setlength\headheight{0in}
\setlength\headsep{0.4in}
\setlength\textheight{8in}
\setlength\textwidth{6in}
\setlength\oddsidemargin{0in}
\setlength\evensidemargin{0in}
\setlength\parindent{0.25in}
\setlength\parskip{0.1in}

\usepackage{amssymb}
\usepackage{amsfonts}
\usepackage{amsmath}
\usepackage{mathtools}
\usepackage{amsthm}

\usepackage{fancyhdr}

\usepackage{enumerate}

      \theoremstyle{plain}
      \newtheorem{theorem}{Theorem}
      \newtheorem{lemma}[theorem]{Lemma}
      \newtheorem{corollary}[theorem]{Corollary}
      \newtheorem{proposition}[theorem]{Proposition}
      \newtheorem{conjecture}[theorem]{Conjecture}
      \newtheorem{question}[theorem]{Question}

      \theoremstyle{definition}
      \newtheorem{definition}[theorem]{Definition}
      \newtheorem{example}[theorem]{Example}
      \newtheorem{game}[theorem]{Game}

      \theoremstyle{remark}
      \newtheorem{remark}[theorem]{Remark}



\pagestyle{fancy}
\renewcommand{\headrulewidth}{0.5pt}
\renewcommand{\footrulewidth}{0pt}
\lfoot{\small \jobname{} -- Updated on \today}
\chead{\small Dr. Clontz -- Fall 2015}
\rfoot{\thepage}
\cfoot{}

\newcommand{\vect}[1]{\mathbf{#1}}
\newcommand{\veci}{\vect i}
\newcommand{\vecj}{\vect j}
\newcommand{\veck}{\vect k}
\newcommand{\<}{\langle}
\renewcommand{\>}{\rangle}
\newcommand{\Arctan}{\textrm{Arctan}}
\newcommand{\p}{\partial}
\newcommand{\mb}{\mathbb}
\newcommand{\sgn}{\textrm{sgn}}
\renewcommand{\div}{\textrm{div}\,}
\newcommand{\curl}{\textrm{curl}\,}
\newcommand{\scurl}{\textrm{scurl}\,}
\newcommand{\dvar}{\,d}

\renewcommand{\labelitemii}{\tiny$\blacksquare$}

\begin{document}

\noindent\textbf{
  MATH 1121 (Calculus for Engineering Technology) Course Outline
}

\section*{1.3 Rectangular Coordinates}

\begin{itemize}
\item Illustrate the following concepts:
  \begin{itemize}
    \item rectangular coordinate system,
    \item \(x\)-axis,
    \item \(y\)-axis,
    \item origin,
    \item quadrants,
    \item coordinates
  \end{itemize}

\item Examples:
  \begin{itemize}
    \item (Example 1) Plot \(A=(2,1)\) and \(B=(-4,-3)\).
    \item (Example 3) Three verticies of a rectangle are
          \(A=(-3,-2)\), \(B=(4,-2)\), \(C=(4,1)\). What is the
          fourth vertex?
  \end{itemize}

\item\textit{
  HW: 1-9, 15-16, 21-24
}
\end{itemize}

\section*{2.1 Some Basic Definitions}

\begin{itemize}
\item Distance Formula
  \begin{itemize}
    \item \(d=\sqrt{(x_2-x_1)^2+(y_2-y_1)^2}\)
    \item (Example 2) Find the distance between \((3,-1)\) and \((-2,-5)\).
  \end{itemize}
\item Slope Formula
  \begin{itemize}
    \item \(m=\frac{y_2-y_1}{x_2-x_1}\)
    \item (Example 3) Find the slope of the line joining
          \((3,-5)\), \((-2,-6)\).
    \item (Example 4) Find the slope of the line joining
          \((3,4)\), \((4,-6)\).
  \end{itemize}
\item Identify parallel/perpendicular lines by slopes.
  \begin{itemize}
    \item Parallel: \(m_1=m_2\)
    \item Perpendicular: \(m_1=-\frac{1}{m_2}\)
    \item (Example 7) Prove that the triangle with vertices
          \(A=(-5,3)\), \(B=(6,0)\), and \(C=(5,5)\) is a right triangle.
  \end{itemize}
\item\textit{
  Suggested HW: 1-20, 29-36
}
\end{itemize}

\section*{2.2 The Straight Line}

\begin{itemize}
\item Point-slope form
  \begin{itemize}
    \item \(y-y_1=m(x-x_1)\)
    \item (Example 2) Find the equation of the line passing through
          \((2,-1)\) and \((6,2)\).
  \end{itemize}
\item Slope-intercept form
  \begin{itemize}
    \item \(y=mx+b\)
    \item (Example 4) Find the slope and \(y\)-intercept of the straight line
          with equation \(2y+4x-5=0\).
  \end{itemize}
\item\textit{
  Suggested HW: 1-21, 33-40
}
\end{itemize}

\section*{2.3 The Circle}

\begin{itemize}
  \item Definition
    \begin{itemize}
      \item A circle is a collection of points equidistant from its center.
    \end{itemize}
  \item Standard form
    \begin{itemize}
      \item \((x-h)^2+(y-k)^2=r^2\)
      \item (Example 1) Sketch \((x-1)^2+(y+2)^2=16\).
      \item (Example 2) Find an equation for the circle with center \((2,1)\)
            which passes through \((4,8)\).
    \end{itemize}
  \item General form
    \begin{itemize}
      \item \(x^2+y^2+Dx+Ey+F=0\)
      \item (Example 4) Find the center and radius of the circle
            \(x^2+y^2-6x+8y-24=0\).
      \item (Example) Find two functions whose graphs represent the circle
            with the previous equation.
    \end{itemize}
  \item\textit{
    HW: 1-32, 37-38
  }
\end{itemize}

\section*{2.4 The Parabola}

\begin{itemize}
  \item Definition
    \begin{itemize}
      \item A parabola is a colleciton of points equidistant from a focus point
        and a directrix line.
      \item (Example 6) Find an equation for the parabola with focus \((2,3)\)
        and directrix \((y=-1)\)
    \end{itemize}
  \item Standard forms with vertex at origin and horizontal/vertical directrix
    \begin{itemize}
      \item \(y^2=4px\) with directrix at \(x=-p\) and focus at \((p,0)\)
      \item \(y^2=-4px\) with directrix at \(x=p\) and focus at \((-p,0)\)
      \item \(x^2=4py\) with directrix at \(y=-p\) and focus at \((0,p)\)
      \item \(x^2=-4py\) with directrix at \(y=p\) and focus at \((0,-p)\)
      \item (Example 2) Find an equation for the parabola with focus
            \((-2,0)\) and directrix \((x=2)\).
      \item (Example 4) Find the focus and directrix of the parabola with
            equation \(2x^2=-9y\).
    \end{itemize}
  \item\textit{
    HW: 1-28
  }
\end{itemize}

\section*{2.5 The Ellipse}

\begin{itemize}
  \item Definition
    \begin{itemize}
      \item An ellipse is a collection of points where the sum of distances
            from two fixed points (called foci) is constant.
      \item The two points furthest/closest apart from each other on an ellipse
            are the endpoints of the major/minor axis.
      \item The sum of distances
            between each point and the foci is the same as the length of
            the major axis. The major axis passes through both foci.
      \item (Example 6) Find an equation for the ellipse with foci at
            \((1,3)\) and \((9,3)\), and a major axis of length \(10\).
    \end{itemize}
  \item Standard form with center at the origin
    \begin{itemize}
      \item \(\frac{x^2}{a^2}+\frac{y^2}{b^2}=1\), with foci given by
            either \((c,0),(-c,0)\) or \((0,c),(0,-c)\) where
            \(a^2+b^2=c^2\)
      \item (Example 3) Sketch the ellipse with equation \(4x^2+16y^2=64\),
            and compute the locations of its foci.
      \item (Example 5) Find the equation of the ellipse centered at the origin
            with an end of its minor axis at \((2,0)\) and containing the point
            \((-1,\sqrt 6)\).
    \end{itemize}
  \item\textit{
    HW: 1-26
  }
\end{itemize}


\section*{Remaining Topics}

\begin{itemize}
  \item 2.6 The Hyperbola
  \item 2.7 Translation of Axes
  \item 1.2 Algebraic Functions
  \item 1.4 The Graph of a Function
% \section*{1.4 The Graph of a Function}

% \begin{itemize}
% \item Definition of a function \(y=f(x)\).
% \item Overview of the table method for plotting function graphs.
% \item Examples:
%   \begin{itemize}
%   \item (Example 1) Graph \(f(x)=3x-5\).
%   \item (Example 3) Graph \(y=3x-5\).
%   \item (Example 4) Graph \(y=\sqrt{x+1}\).
%   \item (Example 6) Graph
%     \[
%       f(x)=
%         \begin{cases}
%           2x+1 & x\leq 1 \\
%           6-x^2 & x>1
%         \end{cases}
%     \]
%   \end{itemize}

% \item\textit{
%   HW: 1-12, 37-40
% }
% \end{itemize}
  \item 3.1 Limits
  \item 3.2 The Slope of a Tangent to a Curve
  \item 3.3 The Derivative
  \item 3.4 The Derivative as an Instantaneous Rate of Change
  \item 3.5 Derivatives of Polynomials
  \item 3.6 Derivatives of Products and Quotients of Functions
  \item 3.7 The Derivative of a Power of a Function
  \item 3.8 Differentiation of Implicit Functions
  \item 3.9 Higher Derivatives
  \item 4.1 Tangents and Normals
  \item 4.4 Related Rates
  \item 4.5 Using Derivatives in Curve Sketching
  \item 4.6 More on Curve Sketching
  \item 4.7 Applied Maximum and Minimum Problems
  \item 4.8 Differentials and Linear Approximations
  \item 5.1 Antiderivatives
  \item 5.2 The Indefinite Integral
  \item 5.3 The Area Under a Curve
  \item 5.4 The Definite Integral
  \item 7.1 The Trigonometric Functions
  \item 7.2 Basic Trigonometric Relations
  \item 7.3 Derivatives of the Sine and Cosine Functions
  \item 7.4 Derivatives of the Other Trigonometric Functions
  \item 8.1 Exponential and Logarithmic Functions
  \item 8.2 Derivative of the Logarithmic Functions
  \item 8.3 Derivative of the Exponentials Function
  \item 9.1 The General Power Formula
  \item 9.2 Basic Logarithmic Form
  \item 9.3 Exponential Form
  \item 9.4 Basic Trigonometric Forms
\end{itemize}

\end{document}