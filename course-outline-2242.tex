\documentclass[11pt]{article}


\pdfpagewidth 8.5in
\pdfpageheight 11in

\setlength\topmargin{0in}
\setlength\headheight{0in}
\setlength\headsep{0.4in}
\setlength\textheight{8in}
\setlength\textwidth{6in}
\setlength\oddsidemargin{0in}
\setlength\evensidemargin{0in}
\setlength\parindent{0.25in}
\setlength\parskip{0.1in}

\usepackage{amssymb}
\usepackage{amsfonts}
\usepackage{amsmath}
\usepackage{mathtools}
\usepackage{amsthm}

\usepackage{fancyhdr}

\usepackage{enumerate}

      \theoremstyle{plain}
      \newtheorem{theorem}{Theorem}
      \newtheorem{lemma}[theorem]{Lemma}
      \newtheorem{corollary}[theorem]{Corollary}
      \newtheorem{proposition}[theorem]{Proposition}
      \newtheorem{conjecture}[theorem]{Conjecture}
      \newtheorem{question}[theorem]{Question}

      \theoremstyle{definition}
      \newtheorem{definition}[theorem]{Definition}
      \newtheorem{example}[theorem]{Example}
      \newtheorem{game}[theorem]{Game}

      \theoremstyle{remark}
      \newtheorem{remark}[theorem]{Remark}



\pagestyle{fancy}
\renewcommand{\headrulewidth}{0.5pt}
\renewcommand{\footrulewidth}{0pt}
\lfoot{\small \jobname{} -- Updated on \today}
\chead{\small Dr. Clontz -- Fall 2015}
\rfoot{\thepage}
\cfoot{}

\newcommand{\vect}[1]{\mathbf{#1}}
\newcommand{\veci}{\vect i}
\newcommand{\vecj}{\vect j}
\newcommand{\veck}{\vect k}
\newcommand{\<}{\langle}
\renewcommand{\>}{\rangle}
\newcommand{\Arctan}{\textrm{Arctan}}
\newcommand{\p}{\partial}
\newcommand{\mb}{\mathbb}
\newcommand{\sgn}{\textrm{sgn}}
\renewcommand{\div}{\textrm{div}\,}
\newcommand{\curl}{\textrm{curl}\,}
\newcommand{\scurl}{\textrm{scurl}\,}
\newcommand{\dvar}{\,d}

\renewcommand{\labelitemii}{\tiny$\blacksquare$}

\begin{document}

\noindent\textbf{
  MATH 2242 (Calculus IV) Course Outline
}
| Vector Calculus (Marsden)

% \section*{1.2 The Inner Product, Length, and Distance}

% \begin{itemize}
%   \item Inner/Dot Product
%     \begin{itemize}
%       \item \(\vect{a}\cdot\vect{b}=a_1b_1+a_2b_2+a_3b_3\)
%     \end{itemize}
%   \item Norm/Magnitude/Length
%     \begin{itemize}
%       \item \(\|\vect{a}\|=\sqrt{\vect{a}\cdot\vect{a}}\)
%       \item Alternate dot product:
%             \(\vect{a}\cdot\vect{b}=\|\vect{a}\|\|\vect{b}\|\cos\theta\)
%     \end{itemize}
%   \item Normalization/Direction
%     \begin{itemize}
%       \item \(\frac{\vect{a}}{\|\vect{a}\|}\)
%     \end{itemize}
%   \item Distance
%     \begin{itemize}
%       \item \(\|\vect{b}-\vect{a}\|\)
%     \end{itemize}
%   \item Inequalities
%     \begin{itemize}
%       \item \(|\vect{a}\cdot\vect{b}|\leq\|\vect{a}\|\|\vect{b}\|\)
%       \item \(\|\vect{a}+\vect{b}\|\leq\|\vect{a}\|+\|\vect{b}\|\)
%     \end{itemize}
% \end{itemize}

% \section*{1.3 Matricies, Determinants, and the Cross Product}

% \begin{itemize}
%   \item Matrices
%     \begin{itemize}
%       \item
%         \(
%           \begin{bmatrix}
%             x_{11} & x_{12} \\
%             x_{21} & x_{22}
%           \end{bmatrix}
%         \)
%       \item
%         \(
%           \begin{bmatrix}
%             x_{11} & x_{12} & x_{13} \\
%             x_{21} & x_{22} & x_{23} \\
%             x_{31} & x_{32} & x_{33}
%           \end{bmatrix}
%         \)
%     \end{itemize}
%   \item Determinants
%     \begin{itemize}
%       \item
%         \(
%           \det\left(\begin{bmatrix}
%             x_{11} & x_{12} \\
%             x_{21} & x_{22}
%           \end{bmatrix}\right)
%             =
%           x_{11}x_{22} - x_{12}x_{21}
%         \)
%       \item
%         \(
%           \det\left(\begin{bmatrix}
%             x_{11} & x_{12} & x_{13} \\
%             x_{21} & x_{22} & x_{23} \\
%             x_{31} & x_{32} & x_{33}
%           \end{bmatrix}\right)
%         \)

%         \(
%             =
%           x_{11}\det\left(\begin{bmatrix}
%             x_{22} & x_{23} \\
%             x_{32} & x_{33}
%           \end{bmatrix}\right) -
%           x_{12}\det\left(\begin{bmatrix}
%             x_{21} & x_{23} \\
%             x_{31} & x_{33}
%           \end{bmatrix}\right) +
%           x_{13}\det\left(\begin{bmatrix}
%             x_{21} & x_{22} \\
%             x_{31} & x_{32}
%           \end{bmatrix}\right)
%         \)
%       \item
%         \(\displaystyle
%           \det(A) = \sum_{i=1}^{n}(-1)^{i+1}x_{1i}\det(A_i)
%         \)
%     \end{itemize}
%   \item Cross-Product
%     \begin{itemize}
%       \item
%         \(
%           \<a_1,a_2,a_3\>\times\<b_1,b_2,b_3\>
%             =
%           \det\left(\begin{bmatrix}
%             \veci & \vecj & \veck \\
%             a_{1} & a_{2} & a_{3} \\
%             b_{1} & b_{2} & b_{3}
%           \end{bmatrix}\right)
%         \)
%       \item
%         \(
%           \|\vect{a}\times\vect{b}\|
%             =
%           \|\vect{a}\|\|\vect{b}\|\sin\theta
%         \)
%       \item
%         \(\vect{a}\), \(\vect{b}\), \(\vect{a}\times\vect{b}\) are mutually
%         orthogonal and follow the right-hand-rule
%     \end{itemize}
%   \item Triple Scalar Product
%     \begin{itemize}
%       \item
%         \(
%           (\vect{a}\times\vect{b})\cdot\vect{c}
%             =
%           \det\left(\begin{bmatrix}
%             \vect{a} \\
%             \vect{b} \\
%             \vect{c}
%           \end{bmatrix}\right)
%         \)
%     \end{itemize}
%   \item Plane Equation
%     \begin{itemize}
%       \item \(\vect{n}\cdot(\vect{x}-\vect{P})=0\)
%       \item \(n_1(x-P_1)+n_2(y-P_2)+n_3(z-P_3)=0\)
%     \end{itemize}
% \end{itemize}

\section*{1.5 \(n\)-Dimensional Euclidean Space}

\begin{itemize}
  \item \(\mb R\), \(\mb R^2\), \(\mb R^3\), \(\mb R^n\)
  \item Addition
    \begin{itemize}
      \item
        \(
          \<x_1,x_2,\dots,x_n\> + \<y_1,y_2,\dots,y_n\>
            =
          \<x_1+y_1,x_2+y_2,\dots,x_n+y_n\>
        \)
    \end{itemize}
  \item Scalar multiplication
    \begin{itemize}
      \item
        \(
          \alpha\<x_1,x_2,\dots,x_n\>
            =
          \<\alpha x_1,\alpha x_2,\dots,\alpha x_n\>
        \)
    \end{itemize}
  \item Inner/Dot Product
    \begin{itemize}
      \item
        \(
          \<x_1,x_2,\dots,x_n\>\cdot\<y_1,y_2,\dots,y_n\>
            =
          \sum_{i=1}^n x_iy_i
        \)
    \end{itemize}
  \item Norm/Length/Magnitude
    \begin{itemize}
      \item
        \(
          \|\vect x\| = (\vect x\cdot\vect x)^{1/2}
        \)
    \end{itemize}
  \item Standard basis vectors
    \begin{itemize}
      \item
        \(
          \vect{e}_1=\<1,0,\dots,0\>
        \),
        \(
          \vect{e}_2=\<0,1,\dots,0\>
        \), \dots,
        \(
          \vect{e}_n=\<0,0,\dots,1\>
        \)
    \end{itemize}
  \item Theorems
    \begin{itemize}
      \item
        \(
          (\alpha\vect{x}+\beta\vect{y})\cdot\vect{z}
            =
          \alpha(\vect{x}\cdot\vect{z}) + \beta(\vect{y}\cdot\vect{z})
        \)
      \item Prove the above theorem.
      \item
        \(
          \vect{x}\cdot\vect{y}
            =
          \vect{y}\cdot\vect{x}
        \)
      \item
        \(
          \vect{x}\cdot\vect{x} \geq 0
        \)
      \item
        \(
          \vect{x}\cdot\vect{x} = 0
        \)
        if and only if
        \(
          \vect{x}=\vect{0}
        \)
      \item
        \(
          |\vect{x}\cdot\vect{y}|
            \leq
          \|\vect{x}\|\|\vect{y}\|
        \)
        (the Cauchy-Schwarz inequality)
      \item Prove the Cauchy-Schwarz inequality.
      \item
        \(
          \|\vect{x}+\vect{y}\|
            \leq
          \|\vect{x}\|+\|\vect{y}\|
        \)
        (the triangle inequality)
      \item Prove the triangle inequality.
    \end{itemize}
  \item Matrices
    \begin{itemize}
      \item
        \(
          A
            =
          \begin{bmatrix}
            a_{11} & a_{12} & \dots  & a_{1n} \\
            a_{21} & a_{22} & \dots  & a_{2n} \\
            \vdots & \vdots & \ddots & \vdots \\
            a_{m1} & a_{m2} & \dots  & a_{mn}
          \end{bmatrix}
        \)
      \item Addition \(A+B\)
      \item Scalar Mutiplication \(\alpha A\)
      \item Transposition \(A^T\)
    \end{itemize}
  \item Vectors as Matrices
    \begin{itemize}
      \item
        \(
          \vect{a}=\<a_1,a_2,\dots,a_n\>
            =
          \begin{bmatrix}
            a_{1}  \\
            a_{2}  \\
            \vdots \\
            a_{n}
          \end{bmatrix}
        \)
      \item
        \(
          \vect{a}^T
            =
          \begin{bmatrix}
            a_{1} & a_{2} & \cdots & a_{n}
          \end{bmatrix}
        \)
    \end{itemize}
  \item Matrix Multiplication
    \begin{itemize}
      \item
        If \(A\) has \(m\) rows and \(B\) has \(n\) columns,
        then \(M=AB\) is an \(m\times n\) matrix.
      \item
        Coordinate \(ij\) of \(M=AB\) is given by
        \(m_{ij}=\vect{a_i}\cdot\vect{b_j}\)
        where \(\vect{a_i}^T\) is the \(i\)th row of \(A\)
        and \(\vect{b_j}\) is the \(j\)th column of \(B\).
      \item
        (Example 4) Compute \(AB\) and \(BA\) for
        \[
          A =
          \begin{bmatrix}
            1 & 0 & 3 \\
            2 & 1 & 0 \\
            1 & 0 & 0
          \end{bmatrix}
        \]
        \[
          B =
          \begin{bmatrix}
            0 & 1 & 0 \\
            1 & 0 & 0 \\
            0 & 1 & 1
          \end{bmatrix}
        \]
      \item
        (Example 5) Compute \(AB\) for
        \[
          A =
          \begin{bmatrix}
            2 & 0 & 1 \\
            1 & 1 & 2
          \end{bmatrix}
        \]
        \[
          B =
          \begin{bmatrix}
            1 & 0 & 2 \\
            0 & 2 & 1 \\
            1 & 1 & 1
          \end{bmatrix}
        \]
    \end{itemize}
  \item Matrices as Linear Transformations
    \begin{itemize}
      \item An \(m\times n\) matrix \(A\) gives a function from \(\mb R^n\)
            to \(\mb R^m\): \(\vect x \mapsto A\vect x\)
      \item This linear transformation satsifies
            \(
              A(\alpha\vect x + \beta\vect y)
                =
              \alpha A\vect x + \beta A\vect y
            \)
      \item (Example 7) Express \(A\vect x\) where \(x=\<x_1,x_2,x_3\>\) and
        \(
          A =
          \begin{bmatrix}
             1 &  0 &  3 \\
            -1 &  0 &  1 \\
             2 &  1 &  2 \\
            -1 &  2 &  2
          \end{bmatrix}
        \).
      \item (Example) Compute where the points
          \((-1,-1,0)\),
          \((0,1,0)\),
          \((1,-1,1)\), and
          \((2,1,1)\)
        in \(\mb R^3\) get
        mapped to in \(\mb R^4\) by \(A\vect x\) from the previous example.
        Then plot the projections of the original points in \(\mb R^3\) onto
        their first two coordinates in \(\mb R^2\),
        and compare this with the projection plot
        of their images in \(\mb R^4\) onto their first two coordinates in
        \(\mb R^2\).
    \end{itemize}
  \item Identity and Inverse
    \begin{itemize}
      \item
        \(
          I_n =
          \begin{bmatrix}
            1      & 0      & \cdots & 0      \\
            0      & 1      & \cdots & 0      \\
            \vdots & \vdots & \ddots & \vdots \\
            0      & 0      & \cdots & 1
          \end{bmatrix}
        \)
      \item
        If \(AA^{-1}=A^{-1}A=I_n\), then \(A\) is invertable and
        \(A^{-1}\) is its inverse.
    \end{itemize}
  \item Determinant
    \begin{itemize}
      \item Let \(A_i\) be the submatrix of \(A\) with the first column
      and \(i\)th row removed. Then
        \(
          \det(A)
            =
          \sum_{i=1}^n
          (-1)^{i+1}
          a_{1i}\det(A_i)
        \)
      \item This is equivalent to
        \(
          \det(A)
            =
          \sum_{\sigma\in S^n}
          \sgn(\sigma)
          \prod_{1\leq i\leq n}
          a_{i,\sigma_i}
        \)
        where \(S^n\) is the collection of all permutations of elements \(1\)
        to \(n\) and \(\sgn(\sigma)\) is \(1\) when \(\sigma\) is obtained by
        an even number of swaps, and \(-1\) when \(\sigma\) is obtained by
        an odd number of swaps.
      \item An \(n\times n\)
        matrix is invertable if and only if its determinant is nonzero.
    \end{itemize}
  \item\textit{
    HW: 1-18, 21-24
  }
\end{itemize}

  % \item 2.1 The Geometry of Real-Valued Functions

\section*{2.3 Differentiation}

\begin{itemize}
  \item Functions \(\mb R^n\to\mb R^m\)
    \begin{itemize}
      \item \(\vect{f}:\mb R^n\to\mb R^m\)
      \item
        \(
          \vect{f}(\vect{x})
            =
          \<f_1(\vect{x}),\dots,f_m(\vect{x})\>
        \) where \(f_i:\mb R^n\to \mb R\)
    \end{itemize}
  \item Partial Derivative Matrix
    \begin{itemize}
      \item
        \(\displaystyle
          \vect{D}\vect{f}(\vect{x})
            =
          \begin{bmatrix}
            \frac{\p f_1}{\p x_1}(\vect x) &
            \cdots &
            \frac{\p f_1}{\p x_n}(\vect x)
            \\
            \vdots & \ddots & \vdots
            \\
            \frac{\p f_m}{\p x_1}(\vect x) &
            \cdots &
            \frac{\p f_m}{\p x_n}(\vect x)
          \end{bmatrix}
        \)
      \item
        We say \(\vect f\) is differentiable at \(\vect{x}\) if
        \(
          \vect{f}(\vect x+\vect h)
            \approx
          \vect{f}(\vect x)+[\vect{D}\vect{f}(\vect{x})]\vect{h}
        \)
        for all \(\vect h\) near \(\vect 0\).
      \item (Example) Let \(\vect f:\mb R^2\to\mb R^2\) be defined by
        \(\vect f(x,y)=\<x^2+y^2,xy\>\), and let
        \(\vect{T}=\vect{D}\vect{f}(1,0)\). Compute
        \(\vect{f}(1.1,-0.1)\) and \(\vect{f}(1,0)+\vect{T}\<0.1,-0.1\>\).
      \item If each \(\frac{\p f_i}{\p x_j}:\mb R^n\to\mb R\)
        is a continuous function near \(\vect x\), then we say \(\vect{f}\)
        is strongly differentiable at \(\vect x\).
    \end{itemize}
  \item Gradient
    \begin{itemize}
      \item If \(f:\mb R^n\to\mb R\), then the gradient vector function
        \(\nabla f:\mb R^n\to\mb R^n\) is defined by
        \(
          \nabla f(\vect x)
            =
          (\vect D f(\vect x))^T
            =
          \<\frac{\p f}{\p x_1}(\vect x),\dots,\frac{\p f}{\p x_n}(\vect x)\>
        \)
      \item
        \(
          [\vect D f(\vect x)]\vect h
            =
          \nabla f(\vect x) \cdot \vect h
        \)
    \end{itemize}
  \item\textit{
    HW: 1-3, 5-21
  }
\end{itemize}

  % \item 2.4 Introduction to Paths and Curves

\section*{2.5 Properties of the Derivative}

\begin{itemize}
  \item Sum/Product/Quotient Rules
    \begin{itemize}
      \item \(\vect D[\alpha \vect f]=\alpha\vect D\vect f\)
      \item \(\vect D[\vect f+\vect g]=\vect D\vect f+\vect D\vect g\)
      \item
        \(
          \vect D[fg]=g\vect Df + f\vect Dg
        \)
      \item
        \(
          \vect D[\frac{f}{g}]
            =
          \frac{
            g\vect Df - f\vect Dg
          }{
            g^2
          }
        \)
      \item Sketch proofs for strongly differentiable \(f,g\).
    \end{itemize}
  \item Chain Rule
    \begin{itemize}
      \item
        \(
          \vect D [\vect f\circ \vect g]
            =
          [\vect D\vect f](\vect g)\vect D\vect g
        \)
      \item
        (Example) Find the rate of change of \(f(x,y)=x^2+y^2\) along
        the path \(\vect c(t)=\<t^2,t\>\) when \(t=1\).
      \item
        (Example 2) Verify the Chain Rule for \(f(u,v,w)=u^2+v^2-w\)
        and \(\vect g(x,y,z)=\<x^2y,y^2,e^{-xz}\>\).
      \item
        (Example 3) Compute \(\vect D[\vect f\circ\vect g](1,1)\) where
        \(\vect f(u,v)=\<u+v,u,v^2\>\) and \(\vect g(x,y)=\<x^2+1,y^2\>\).
    \end{itemize}
  \item\textit{
    HW: 6-13, 15-16
  }
\end{itemize}

  % \item 2.6 Gradients and Directional Derivatives

\section*{3.2 Taylor's Theorem}

\begin{itemize}
  \item First-Order Taylor Formula
    \begin{itemize}
      \item
        \(
          f(\vect x + \vect h)
            \approx
          f(\vect x) + [\vect Df(\vect x)]\vect h
        \)
      \item Alternate form:
        \(
          f(\vect x + \vect h)
            \approx
          f(\vect x)
            +
          \sum_{i=1}^{n}\frac{\p f}{\p x_i}(\vect x)h_i
        \)
    \end{itemize}
  \item Second-Order Taylor Formula
    \begin{itemize}
      \item
        \(
          f(\vect x + \vect h)
            \approx
          f(\vect x)
            +
          \sum_{i=1}^{n}\frac{\p f}{\p x_i}(\vect x)h_i
            +
          \frac{1}{2}
          \sum_{i,j=1}^{n}\frac{\p^2 f}{\p x_i\p x_j}(\vect x)h_ih_j
        \)
      \item
        (Example 3) Find linear and quadratic functions of \(x,y\) which
        approximate \(f(x,y)=\sin(xy)\) near the point \((1,\pi/2)\).
    \end{itemize}
  \item\textit{
    HW: 1-12
  }
\end{itemize}

  % \item 4.1 Acceleration and Newton's Second Law
  % \item 4.2 Arc Length

\section*{4.3 Vector Fields}

\begin{itemize}
  \item Vector Fields
    \begin{itemize}
      \item A vector field is a map \(f:\mb R^n\to\mb R^n\) assinging an
        \(n\)-dimensional vector to each point in \(\mb R^n\)
      \item (Example 1) The velocity field of a fluid may be modeled as a
        vector field.
      \item (Example 2) Sketch the rotary motion given by the vector
        field \(\vect V(x,y)=\<-y,x\>\).
    \end{itemize}
  \item Gradient Vector Fields
    \begin{itemize}
      \item \(\nabla f = \<\frac{\p f}{\p x_1},\dots,\frac{\p f}{\p x_n}\>\)
      \item (Example) The derivative of a scalar function
        \(f:\mb R^n\to \mb R\) in the direction given by a unit vector
        \(\vect v\)
        is given by \(\nabla f\cdot \vect v\). Show that the maximum value
        of a directional derivative for a fixed point
        is given by \(\|\nabla f\|\)
        and attained by the direction \(\frac{1}{\|\nabla f\|}\nabla f\).
      \item (Example 4) If temperature is given by \(T(x,y,z)\), then the
        energy or heat flux field is given by \(\vect J = -k\nabla T\) where
        \(k\) is the conductivity of the body. Level sets are called isotherms.
      \item (Example 5) The gravitational potential of bodies with mass \(m,M\)
        is given by \(V=-\frac{mMG}{r}\) where \(G\) is the gravitational
        constant and \(r\) is the distance between the bodies, and the
        gravitational force field is given by \(\vect F=-\nabla V\).
        Show that \(\vect F = -\frac{mMG}{r^3}\vect r\), where \(\vect r\)
        is the vector pointing from the center of mass \(M\) to the center
        of mass \(m\).
      \item A vector field \(\vect F:\mb R^n\to\mb R^n\) is conservative iff
        there exists
        a potential function \(f:\mb R^n\to\mb R\) such that
        \(\vect F=\nabla f\).
      \item (Example) Show that \(\vect W=\<2y+1,2x\>\) is conservative.
      \item (Example 7) Show that \(\vect V=\<y,-x\>\) is not conservative.
    \end{itemize}
  \item Flow Lines
    \begin{itemize}
      \item A flow line for a vector field \(\vect F:\mb R^n\to\mb R^n\)
        is a path \(\vect c:\mb R\to\mb R^n\) satisfying
        \(\vect c'(t)=\vect F(\vect c(t))\).
      \item (Example 8) Show that \(\vect c(t)=\<\cos t,\sin t\>\) is a flow
        line for \(\vect F=\<-y,x\>\), and find some other flow lines.
    \end{itemize}
  \item\textit{
    HW: 1-22
  }
\end{itemize}


\section*{Remaining Topics}

\begin{itemize}
  \item 4.4 Divergence and Curl
  \item 5.3 The Double Integral Over More General Regions
  \item 5.4 Changing the Order of Integration
  \item 5.5 The Triple Integral
  \item 6.1 The Geometry of Maps from \(\mb R^2\) to \(\mb R^2\)

% \section*{1.4 Cylindrical and Spherical Coordinates}

% \begin{itemize}
%   \item Transformation of variables
%   \item Polar Coordinates
%     \begin{itemize}
%       \item \(p(r,\theta)=(r\cos\theta,r\sin\theta)\)
%       \item Usually, assume \(r\geq0\) and \(0\leq\theta\leq2\pi\)
%       \item \(r^2=x^2+y^2\), \(\theta=\Arctan(\frac{y}{x})+k\pi\)
%     \end{itemize}
%   \item Cylindrical Coordinates
%     \begin{itemize}
%       \item \(c(r,\theta,z)=(r\cos\theta,r\sin\theta,z)\)
%       \item Usually, assume \(r\geq0\) and \(0\leq\theta\leq2\pi\)
%       \item \(r^2=x^2+y^2\), \(\theta=\Arctan(\frac{y}{x})+k\pi\), \(z=z\)
%       \item (Example 1) Convert \(A=c(8,2\pi/3,-3)\) from cylindrical to
%             Cartesian. Convert \(B=(6,6,8)\) from Cartesian to cylindrical.
%             Plot both.
%     \end{itemize}
%   \item Spherical Coordinates
%     \begin{itemize}
%       \item \(
%               s(\rho,\theta,\phi)
%                 =
%               (
%                 \rho\sin\phi\cos\theta,
%                 \rho\sin\phi\sin\theta,
%                 \rho\cos\phi
%               )
%             \)
%       \item Usually, assume \(\rho\geq0\), \(0\leq\theta\leq2\pi\), and
%             \(0\leq\phi\leq\pi\)
%       \item \(\rho^2=x^2+y^2+z^2\),
%             \(\theta=\Arctan(\frac{y}{x})+k\pi\),
%             \(\phi=\Arctan(\frac{\sqrt{x^2+y^2}}{z})\)
%       \item (Example 2) Convert \(A=(1,-1,1)\) from Cartesian to spherical.
%             Convert \(B=s(3,\pi/6,\pi/4)\) from spherical to Cartesian.
%             Convert \(C=(2,-3,6)\) from Cartesian to spherical.
%             Convert \(D=s(1,-\pi/2,\pi/4)\) from spherical to Cartesian.
%             Plot all four.
%       \item (Example 3) Express the equations \(xz=1\) and \(x^2+y^2-z^2=1\)
%             in terms of \(\rho,\phi,\theta\).
%     \end{itemize}
%   \item\textit{
%     HW: 1-16
%   }
% \end{itemize}
  \item 6.2 The Change of Variables Theorem
  \item 7.1 The Path Integral
  \item 7.2 Line Integrals
  \item 7.3 Parametrized Surfaces
  \item 7.4 Area of a Surface
  \item 7.5 Integrals of Scalar Functions Over Surfaces
  \item 7.6 Surface Integrals of Vector Fields
  \item 8.1 Green's Theorem
  \item 8.2 Stokes' Thoerem
  \item 8.3 Conservative Fields
  \item 8.4 Gauss' Theorem
\end{itemize}

\end{document}